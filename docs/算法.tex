\documentclass[11pt, a4paper]{article}
\usepackage[top=2cm, bottom=2cm, left=2cm, right=2cm]{geometry}  

\usepackage{xeCJK} % 设置中文显示
\setCJKmainfont{CESI_FS_GB18030}

\usepackage{caption}
\captionsetup{font={footnotesize}}

\usepackage{algorithm}
\usepackage{algorithmic}
\usepackage{latexsym}
\usepackage{amsmath}
\renewcommand{\algorithmicrequire}{\textbf{Input:}} 
\renewcommand{\algorithmicensure}{\textbf{Output:}}
% \begin{algorithm}[htbp]
%     \caption{Rabin-Karp(T,P,d,p)}
%     \label{alg:rabin-karp}
%     \begin{algorithmic}[1] % 显示行号
%         \STATE $n \leftarrow length[T]; m \leftarrow length[P]$
%         \STATE $h \leftarrow d \mod p$
%         \STATE $p \leftarrow 0; t_0 \leftarrow 0$
%         \STATE $ans \leftarrow list()$
%         \FOR{$i \leftarrow 1$ to $m$}
%             \STATE $p \leftarrow (dp + P[i]) \mod q$
%             \STATE $t_0 \leftarrow (dt_0 + T[i]) \mod q$
%         \ENDFOR
%         \FOR{$s \leftarrow 0$ to $n-m$}
%             \IF{$p = t_s$}
%                 \IF{$P[0 \dots m-1]=T[s \dots s+m-1]$}
%                     \STATE ans.append($s$)
%                 \ENDIF
%             \ENDIF
%             \IF{$s<n-m$}
%                 \STATE $t_{s+1} \leftarrow (d(t_s-T[s+1]h)+T[s+m+1]) \mod q$
%             \ENDIF
%         \ENDFOR
%     \end{algorithmic}
% \end{algorithm}

\begin{document}
  %\begin{algorithm}
  %  \caption{Algorithmn of Perceptron}
  %  \begin{algorithmic}[1] % 每行显示行号
  %    \REQUIRE
  %    \ENSURE
  %  \end{algorithmic}
  %\end{algorithm}
  \begin{algorithm}
    \caption{Algorithmn of Perceptron}
    \begin{algorithmic}[1] % 每行显示行号
      \REQUIRE $T=\{(x_1,y_1),\dots,(x_N,y_N)\},\text{where }x_i \in \mathcal{X} = R^n,y_i \in \mathcal{Y} = \{-1, +1\}; \text{learning rate }\eta$
      \ENSURE $w, b$
      \STATE $w \leftarrow 0 \in R^n; b \leftarrow 0$
      \WHILE {True}
        \FOR {$(x_i,y_i)$ in $T$}
          \IF {$y_i(w \dot x_i) \leq 0$}
            \STATE $w \leftarrow w + \eta y_i x_i$
            \STATE $b \leftarrow b + \eta y_i$
          \ENDIF
        \ENDFOR
        \IF {no sample in $T$ is misclassified}
          \STATE break
        \ENDIF
      \ENDWHILE
    \end{algorithmic}
  \end{algorithm}
  实例点$(x_i,y_i)$被用于更新的次数越多, 意味着它离超平面越接近, 也就越难以正确分类. 这样的实例对学习结果影响最大
\end{document}
